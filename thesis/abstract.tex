\begin{abstract}

An interactive argument (or computationally sound proof system) is a relaxation of an interactive proof, introduced in \cite{10.1145/22145.22178}. The difference is that the prover is restricted to be a polynomial-time algorithm for an interactive argument, whereas no such restrictions on the prover apply for an interactive proof. In this thesis, we focus on the practical usage of \textbf{linear time} interactive argument with or without zero-knowledge property. Additionally, many interactive argument systems use polynomial commitment as a fundamental building block. And polynomial commitment can imply a general interactive argument system. Hence, we will investigate the linear time polynomial commitment instead.

Zero-knowledge proofs allow an untrusted prover to convince a sceptical verifier that a statement is true without revealing any further information about why the statement is true. Example use-cases include verifiable computing, where a powerful, but untrusted server proves, to a computationally weak client, that they performed a large calculation correctly. After years of research improving the proof size and verifier run-time of zero-knowledge proofs, prover runtime remains a major bottleneck.

A line of work \cite{brakedown} \cite{cryptoeprint:2020/1426} \cite{BCL22} attempts to address this with zero-knowledge protocols where prover runtime is a constant multiplied by the time taken to perform the calculation. The only one of these works which investigates the practical efficiency of their constructions is \cite{brakedown}, which makes stronger security assumptions than the other works to achieve zero-knowledge and good verifier runtime. On the other hand, \cite{cryptoeprint:2020/1426} \cite{BCL22} rely on specialised constructions of error correcting codes, hash functions, and sub-protocols whose practical performance is unknown.

The protocols in these works rely on special families of linear-time error-correcting codes, whose properties influence the final performance of the proof systems. Firstly, \cite{brakedown} \cite{cryptoeprint:2020/1426} \cite{BCL22} rely on codes with a tensor structure. The lower the dimension of the tensors, the smaller the proof size and verification time of the zero-knowledge proofs. The implementation of \cite{brakedown} uses low dimension. The first goal of this project is to extend the implementation of \cite{brakedown} to higher dimension and investigate the impact of dimension on performance, at various different security levels.

Secondly, while \cite{brakedown} rely on strong assumptions in order to achieve zero-knowledge, \cite{cryptoeprint:2020/1426} \cite{BCL22} use codes with extra zero-knowledge properties. Codes with zero-knowledge properties are easy to obtain from plain codes via simple transformations that is similar to one-time-pad encryption, and codes with stronger zero-knowledge properties use more complex transformations, as described in \cite{10.1145/2554797.2554815}. The second goal of this project is to investigate the performance impact of adding zero-knowledge to the \cite{brakedown} polynomial commitment scheme using zero-knowledge codes, using the plain codes implemented by \cite{brakedown} as a starting point, and assess which transformations are more practical.

\end{abstract}
