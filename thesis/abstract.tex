\begin{abstract}

Interactive zero-knowledge proofs allow an untrusted prover to convince a skeptical verifier that a statement is true without revealing any
further information about why the statement is true. The polynomial commitment scheme is the key component of many proof systems, which
allows the prover to commit to a polynomial, and later reveal the evaluation of the polynomial at a given point, while allowing the verifier can
check that the evaluation is correct. After years of research, many
schemes with linear prover time have been proposed, however, there is
little work on their concrete efficiency and performance.

In this thesis, we want to investigate the concrete efficiency of those
polynomial commitment schemes, especially in high-dimensional situations. We implement the protocol in Rust, benchmark the performance and analyze the result. Additionally, we would investigate various ways to add zero-knowledge property into the polynomial commitment scheme and research their advantages and limitations.

We conclude that the efficiency and the soundness error of high dimensional polynomial commitment schemes are not acceptable to be
used in practice because of the lack of linear code that is both efficient in encoding and provides a large relative distance guarantee. And
we can add zero-knowledge property into the polynomial commitment
schemes at the cost of increasing prover time, verifier time, and proof
size by roughly a factor of two. We also implement and benchmark the performance of the LWE protocol.

\end{abstract}
