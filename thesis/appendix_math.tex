\chapter{Mathematical Facts}

% https://www.youtube.com/watch?v=N9L3oGMK3mc
\begin{lemma}
\label{lemma:nchoosem}

$\binom{n}{m} \le (\frac{en}{m})^m$, for $n, m \in \mathbb{Z}^+$

\end{lemma}
\begin{proof}
\begin{align}
\log m! 
    &= \sum_{i=1}^{m} \log i \nonumber \\
    &\ge \int_1^m \log x \, dx \nonumber \\
    &= [x \log x - x]_1^m \nonumber \\
    &= m \log m - m + 1 \label{eq:logm!}
\end{align}
\begin{align}
m! 
    &= e^{\ln m!} \nonumber \\
    &\ge e^{m \log m - m + 1} \nonumber 
    && \text{apply equation \ref{eq:logm!}} \\
    &= e^{\log m^m} \cdot e^{-m} \cdot e \nonumber \\
    &= m^m \cdot e^{-m} \cdot e \nonumber \\
    &= (\frac{m}{e})^m \cdot e \nonumber \\
    &\ge (\frac{m}{e})^m \label{eq:m!}
\end{align}
\begin{align}
\binom{n}{m} 
    &= \frac{n \cdot (n-1) \cdot (n-2) \cdots (n-m+1)}{m!} \nonumber \\
    &\le \frac{n^m}{m!} \nonumber \\
    &\le \frac{n^m}{(\frac{m}{e})^m} \nonumber 
    && \text{apply equation \ref{eq:m!}} \\
    &= (\frac{en}{m})^m
\end{align}
\end{proof}

% https://math.stackexchange.com/questions/544667/upper-bound-for-1-1-xx
\begin{lemma}
\label{lemma:(1-1x)x}

$(1 - \frac{1}{x})^x \le \frac{1}{e}$, for $x \ge 1$

\end{lemma}

\begin{proof}

\begin{align}
\shortintertext{Recall that for $x \in \mathbb{R}$}
    1 + x &\le e^x \nonumber \\
\shortintertext{Then for $x \in \mathbb{R}$}
    1 - x &\le e^{-x} \nonumber \\
\shortintertext{Then for $x \neq 0$}
    1 - \frac{1}{x} &\le e^{-\frac{1}{x}} \nonumber \\ 
\shortintertext{And, since $t \mapsto t^x$ is increasing on $[0, \infty]$ for $x \ge 1$}
   (1 - \frac{1}{x})^x &\le \frac{1}{e}
\end{align}

\end{proof}

% https://www.doubtnut.com/question-answer/prove-that-maximum-value-of-1-xx-is-e1-e-4556542
\begin{lemma}
\label{lemma:(1x)x}

$(\frac{a}{x})^x \le e^\frac{a}{e}$, for $x > 0$, $a > 0$

\end{lemma}

\begin{proof}

\begin{align}
\shortintertext{Let $f(x) = (\frac{a}{x})^x $}
    \ln f(x) &= x \cdot \ln (\frac{a}{x}) \nonumber = - x \cdot \ln \frac{x}{a} \\
\shortintertext{Take derivative from both sides}
    \frac{1}{f(x)} \frac{df(x)}{dx} &= -\ln \frac{x}{a} - x \cdot \frac{a}{x} \cdot \frac{1}{a} = -\ln \frac{x}{a} - 1 \nonumber \\
    \frac{df(x)}{dx} &= - f(x) \cdot (\ln \frac{x}{a} + 1) \nonumber = - (\frac{a}{x})^x \cdot (\ln \frac{x}{a} + 1) \\
\shortintertext{Let $\frac{df(x)}{dx} = 0$}
    - (\frac{a}{x})^x \cdot (\ln \frac{x}{a} + 1) &= 0 \nonumber \\
    (\ln \frac{x}{a} + 1) &= 0 \nonumber \\
    x &= \frac{a}{e} \nonumber \\
\shortintertext{$\frac{df(x)}{dx} > 0$ when $x < \frac{a}{e}$, and $\frac{df(x)}{dx} < 0$ when $x > \frac{a}{e}$}
\shortintertext{Therefore, $x = \frac{a}{e}$ is a maximum point}
    (\frac{a}{x})^x &= f(x) \le f(\frac{a}{e}) = {e}^\frac{a}{e}
\end{align}

\end{proof}

