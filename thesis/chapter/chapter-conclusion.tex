\chapter{Conclusion}

In this thesis, we investigated the concrete efficiency of polynomial commitment schemes. We implement the protocol in Rust, benchmark the performance and analyze the result.
According to the benchmark result, as the dimension increases, it generally requires more time to complete the commit phase for the prover, and less time is required to complete the verification phase for the verifier. Also, the polynomial commitment scheme is highly parallelizable. Running the polynomial commitment scheme with the same setting, using
8-threads-parallelism can provide approximately a 4x speedup. However, the soundness error in the high-dimensional situation is really bad, making the protocol unusable in practice. The soundness error can be decreased by either increasing the number of tested tuples or by increasing the relative distance of the underlying linear code. However, the soundness error is not sensitive to the number of tested tuples and the length of the code is usually quite limited. Therefore, one potential solution to this problem is finding a linear code with large relative distance property.


Additionally, we investigated various ways to add zero-knowledge property into the polynomial commitment scheme and research their advantages and limitations. 
We begin by investigating the method in \cite{10.1145/2554797.2554815}, which is originally proposed to boost relative distance. Later researchers have found that it provides zero-knowledge property as well. However, since the encoding function after this transformation is too slow, we conclude that using the method to transform a normal linear code to a zero-knowledge linear code is not practical. Then we tried to add the zero-knowledge property by using a random pad, which is similar to one-time-pad encryption.
The cost is increasing prover time, verifier time, and proof size by roughly a factor of two. Later we find that this construction alone is not enough to guarantee zero-knowledge property. We need the linear code to satisfy the $l$-query independent property to prevent information leaking. However, how to prove a given linear code satisfies this property is an open and difficult problem. One future research direction is to prove a specific family of linear code satisfies this property. For example, one may try to prove the Brakedown linear code with the message length of $k$ is roughly $\frac{k}{4}$-query independent.
