\chapter{Polynomial Commitment}

In this chapter, we present a general polynomial commitment scheme in the language of IOP for arbitrary dimension $t$. The scheme is an extension of the polynomial commitment scheme for $t=2$ described in \cite{brakedown}. We first extend the scheme to the $t=3$ situation so that readers can have a good intuition on how it works. Then we generalize it to arbitrary $t$ with detailed analysis available.

\section{Notation}

Let $g$ be a multilinear polynomial with $n$ coefficients. For simplicity we assume that $n = m^t$ for some integer $m$. And let $u$ denote the coefficient vector of $g$ in the Lagrange basis, which means $u$ represents all evaluations of $g$ over inputs in hypercube $\{0, 1\}^{\log n}$. 
We can rearrange $u$ to be a $\underbrace{n^{\frac{1}{t}} \times n^{\frac{1}{t}} \times \cdots \times n^{\frac{1}{t}}}_{t \text{ times}}$ matrix, such that we can index entries in this matrix easily by elements from set $[m]^t$.

Let $N = \rho^{-1} \cdot m$ and \text{Enc}: $\mathbb{F}^m \rightarrow \mathbb{F}^N$ represent the encoding function of a linear code with a constant rate $\rho > 0$ and a constant minimum relative distance $\gamma > 0$.

Let $\text{Enc}_i(M)$ denote the function that encode every stripes in the $i$th dimension of matrix $M$ using encoding function Enc. For example, $\text{Enc}_1(M)$ will encode each column of a $n \times n $ matrix and produce a $N \times m$ matrix.

\begin{lemma}[Polynomial Evaluation \cite{brakedown}]
\label{lemma:petq}

For an $l$-variate multilinear polynomial $g$ represented in the Lagrange basis via a vector $u \in \mathbb{F}^{n}$ where $2^l = n$, given an evaluation point $x \in \mathbb{F}^l$, $g(x)$ can be evaluated using the following tensor product identity: 

\[
    g(x) = \langle (x_1, 1-x_1) \otimes (x_2, 1-x_2) \otimes \cdots \otimes (x_l, 1-x_l) , u \rangle
\]

And for any $ 1 \le t  \le l$, there always exist vectors $q_1, q_2, \cdots , q_t \in \mathbb{F}^{n^{\frac{1}{t}}}$ such that the following holds:

\[
    (x_1, 1-x_1) \otimes (x_2, 1-x_2) \otimes \cdots \otimes (x_l, 1-x_l) = q_1 \otimes q_2 \otimes \cdots \otimes q_t
\]

\end{lemma}

\section{Polynomial Commitment for t = 3}

\textbf{Commitment Phase.}

Let $M_0 = u$ and $M_0^{\prime} = \text{Enc}_1(\text{Enc}_2(M_0)) \in \mathbb{F}^{N \times N \times m}$. Send $M_0^{\prime}$ to the verifier.

\textbf{Testing Phase.}

The testing phase consists of $2$ rounds, with each round reducing the dimension by 1.

In round 1, the verifier will send a random value $r_1 \in \mathbb{F}^m$ to the prover.
The prover will compute a linear combination $M_1 \in \mathbb{F}^{m \times m}$ of the 3rd dimension of matrix $M_0$.
Namely, $M_1[i,j] = \sum_{k=1}^{m} r_1[k] \cdot M_0[i,j,k]$ for $1 \le i, j \le m$. 
Let $M_1^\prime = \text{Enc}_1(M_1) \in \mathbb{F}^{N \times m}$.
Then the prover sends $M_1^\prime$ to the verifier.

In round 2, the verifier will send a random value $r_2 \in \mathbb{F}^m$ to the prover.
The prover will compute a linear combination $M_2 \in \mathbb{F}^{m}$ of the 2nd dimension of matrix $M_1$.
Namely, $M_2[i] = \sum_{k=1}^{m} r_2[k] \cdot M_1[i, k]$. 
Let $M_2^\prime = M_2 \in \mathbb{F}^{m}$.
Then the prover sends $M_2^\prime$ to the verifier.

Then the verifier will perform a probabilistic check to make sure $M_0^\prime $, $ M_1^\prime$ and $M_2^\prime$ are consistent with each other. Formally speaking, the verifier will sample $l$ random tuple $(j_1, j_2, j_3)$ from space $[N] \times [N] \times [N]$. 
For each tuple $(j_1, j_2, j_3)$, let $M_1^\prime[j_1, *]$ denotes the $j_1$-th row in $M_1^\prime$.
The verifier will check whether the following equation holds:

$$
    \text{Enc}(M_1^\prime[j_1, *])[i_2] \stackrel{?}{=} \sum_{k=1}^m r_1[k] \cdot M_0^{\prime}[j_1,j_2,k]
$$
$$
    \text{Enc}(M_2^\prime[*])[j_1] \stackrel{?}{=} \sum_{k=1}^m r_2[k] \cdot M_1^{\prime}[j_1,k]
$$

\textbf{Evaluation Phase.}

Let $q_1, q_2, q_3 \in \mathbb{F}^{m}$ be vectors such that $g(x) =\langle q_1 \otimes q_2 \otimes q_3, u \rangle $. The evaluation phase is identical to the testing phase, except that in round $i$, the random value $r_i$ is replaced by $q_i$. If all consistency checks passed, then the verifier outputs $M_{3}$ as $g(x)$.

\section{Polynomial Commitment for Arbitrary t}

\subsection{Protocol}

\textbf{Commitment Phase.}

Let $M_0 = u$ and $ M_0^{\prime} = \text{Enc}_1 \circ \text{Enc}_2 \circ \cdots \circ \text{Enc}_{t-1} (M_0) \in \mathbb{F}^{\overbrace{N \times N \times \cdots \times N}^{t-1 \text{ times}} \times m}$. Send $M_0^{\prime}$ to the verifier.

\textbf{Testing Phase.}

The testing phase consists of $t-1$ rounds, with each round reducing the number of dimensions by 1.

In round i, the verifier will send a random value $r_i \in \mathbb{F}^m$ to the prover.
The prover will compute a linear combination $M_i \in \mathbb{F}^{\overbrace{m \times m \times \cdots \times m}^{t-i \text{ times}}}$ of the last dimension of matrix $M_{i-1}$.
Namely, for $1 \le j_1,j_2, \cdots, j_{t-i} \le m$:
$$M_i[j_1,j_2, \cdots, j_{t-i}] = \sum_{k=1}^{m} r_{i}[k] \cdot M_{i-1}[j_1,j_2, \cdots, j_{t-i}, k]$$
Let 
$$
M_i^\prime = \text{Enc}_1 \circ \text{Enc}_2 \circ \cdots \circ \text{Enc}_{t - i - 1}(M_i)\in \mathbb{F}^{\overbrace{N \times N \times \cdots \times N}^{t-i-1 \text{ times}} \times m}
$$
Then the prover sends $M_i^\prime$ to the verifier.

Then the verifier will perform a probabilistic check to make sure 
$M_0^\prime$, $M_1^\prime$, $M_2^\prime, \cdots, M_{t-1}^\prime$ 
are consistent with each other. Formally speaking, the verifier will sample $l$ random tuple $(j_1, j_2, \cdots, j_t)$ from space $\overbrace{[N] \times [N] \times \cdots \times [N]}^{t \text{ times}}$. 
For each tuple $(j_1, j_2, \cdots, j_t)$,
the verifier will check whether the following equation holds for every $i \in [t-1]$:

$$
    \text{Enc}(M_i^\prime[j_1, j_2, \cdots, j_{t-i-1}, *])[j_{t-i}] \stackrel{?}{=} \sum_{k=1}^m r_i[k] \cdot M_{i-1}^{\prime}[j_1,j_2, \cdots, j_{t-i},k]
$$

\textbf{Evaluation Phase.}

Let $q_1, q_2, \cdots, q_t \in \mathbb{F}^{m}$ be vectors such that $g(x) =\langle q_1 \otimes q_2 \otimes \cdots \otimes q_t, u \rangle $. The evaluation phase is identical to the testing phase, except that in round $i$, the random value $r_i$ is replaced by $q_i$. If all consistency checks passed, then the verifier outputs $M_{t}^{\prime}$ as $g(x)$.

\section{Analysis}

We refer to the result in \cite{cryptoeprint:2020/1426} and summarize to the following lemmas. 

\begin{lemma}
\label{lemma:pc-completeness}
The testing phase (proximity test) has perfect completeness.
\end{lemma}

\begin{lemma}
\label{lemma:pc-soundness}
The testing phase (proximity test) has soundness error:
$$
    \epsilon(\Delta_\otimes, t, l) = \frac{d(d^t-1)}{4(d-1)|\mathbb{F}|} + (1 - \text{ min}\{\frac{\delta^t}{4}, \Delta_\otimes \})^l
$$
where $d = \delta \cdot N$, and $\delta$ denotes the relative distance.
\end{lemma}



\section{Implementation Details}

\subsection{Merkle Tree Commitment}


A Merkle Tree is a data structure that allows one to commit to $l = 2^{d}$ messages by a single hash value $h$, such that revealing any bit of the message require $d+1$ hash values. A Merkle hash tree is presented by a binary tree of depth $d$ where $l$ messages elements $m_1, m_2, \cdots, m_l$ are assigned to the leaves of the tree. The values assigned to internal nodes are computed by hashing the value of its two child nodes. To reveal $m_i$, we need to reveal $m_i$ together with the values on the path from $m_i$ to the root. We denote the algorithm as follows:

\begin{enumerate}
    \item $h \leftarrow \textsc{Merkle.Commit}(m_1, m_2, \cdots, m_l)$
    \item $(m_i, \pi_i) \leftarrow \textsc{Merkle.Open}(m, i)$
    \item $\{\textsc{accept}, \textsc{reject}\} \leftarrow \textsc{Merkle.Verify}(\pi_i, m_i, h)$
\end{enumerate}

Coefficient matrices $M_0^\prime, M_1^\prime, \cdots, M_{t-1}^\prime$ sent by the prover may be replaced by a Merkle tree commitment to that matrix, and each query the verifier makes to a matrix is answered by the prover with Merkle tree authentication path for the answer.

And since each time the verifier will query a strip of elements in a matrix (i.e. $M_i^\prime[i_1, i_2, \cdots, i_{t-i-1}, *]$), it is possible to zip such a strip of elements into a single node in Merkle-tree's leaf level to decrease runtime complexity and communication complexity.

\subsection{Parallelism}

Most of the computations for the commitment scheme can be done in parallel in a natural fashion. There is little data dependence among them. Therefore, it is possible to run the commitment scheme using multiple threads to increase efficiency significantly for both the prover and the verifier.


\section{Benchmark}

\subsection{Runtime}


\begin{table}[h!]
\centering
\begin{tabular}{| c | m{4em}  | m{3em}  | m{3.5em} | m{2.5em} | m{5em} | m{7em} |} 
 \hline
 Dimension & Message Length & Code Length & Commit Time [ms] & Verify Time [ms] & Soundness Error & Communication Complexity [Field Element] \\ [0.5ex] 
 \hline\hline
 2 & 1024   & 1762 & 41737  & 3057  & 0.37 & 1206579 \\
 \hline
 3 & 101    & 174 & 99642  & 623  & 1.76 & 235621  \\
 \hline
 4 & 32     & 56 & 153558  & 204  & 1.98 & 114701   \\
 \hline
\end{tabular}
\caption{Runtime of polynomial commitment scheme with $2^{20}$ coefficients, 1 threads, linear code with relative distance 0.07, and 1000 test tuples.}
\label{table:benchmark-pc-1}
\end{table}


We benckmark the above polynomial commitment scheme on a computer with
Intel \textregistered \, Core  \textsuperscript{TM} i7-7700HQ CPU @ 2.80GHz (Kabylake), L1 cache: 128KB, L2 cache: 256KB and L3 cache: 6MB. There are 8 physical CPU cores available on this machine. The runtimes are summarized in the table \ref{table:benchmark-pc-1} and table \ref{table:benchmark-pc-2}.


Running the polynomial commitment scheme with the same setting, using 8-threads-parallelism can provides approximately a 4x speedup.

\begin{table}[h!]
\centering
\begin{tabular}{| c | m{4em}  | m{3em}  | m{3.5em} | m{2.5em} | m{5em} | m{7em} |} 
 \hline
 Dimension & Message Length & Code Length & Commit Time [ms] & Verify Time [ms] & Soundness Error & Communication Complexity [Field Element] \\ [0.5ex] 
 \hline\hline
 2 & 1024   & 1762 & 10048 & 776 & 0.37 & 1206579  \\
 \hline
 3 & 101    & 174 & 24314 & 165 & 1.76 & 235621 \\
 \hline
 4 & 32     & 56 & 37961 & 63 & 1.98 & 114701  \\ 
 \hline
\end{tabular}
\caption{Runtime of polynomial commitment scheme with $2^{20}$ coefficients, 8 threads, linear code with relative distance 0.07, and 1000 test tuples.}
\label{table:benchmark-pc-2}
\end{table}


As the dimension increases, it is generally require more time to complete the commit phase for the prover. And less time is required to complete the verify phase for the verifier. Also high dimensional polynomial commitment scheme will have less communication complexity. However, since the relative distance is decreasing as the tensor code's dimension is increasing, the soundness error will also increase. In fact, the soundness error for 3-dimensional and 4-dimensional polynomial commitment scheme is higher than 1, which is unusable in practice.

\subsection{Soundness Error}

According to lemma \ref{lemma:pc-soundness}, we can compute the soundness error summarized in the table \ref{table:benchmark-pc-3}.

\begin{table}[h!]
\centering
\begin{tabular}{| c | m{5em} | m{5em} | m{5em}  | m{5em}|} 
 \hline
 Dimension & Number of Test Tuples & Code Length & Code Relative Distance & Soundness Error \\ [0.5ex] 
 \hline\hline
 
 \multirow{3}{*}{2} & 100 & 1762 & 0.07 & 1.66  \\
  & 1000 & 1762 & 0.07 & 0.37  \\
  & 100 & 1762 & 0.55* & 0.0003  \\
 \hline
 
 \multirow{3}{*}{3} & 100 & 174  & 0.07 & 1.97 \\
 & 1000 & 174  & 0.07 & 1.76 \\
 & 100 & 174  & 0.55* & 0.01 \\
 \hline
 
 \multirow{3}{*}{4} & 100 & 56   & 0.07 & 1.99  \\
  & 1000 & 56   & 0.07 & 1.98  \\
  & 100 & 56   & 0.55* & 0.10  \\ 
 \hline
\end{tabular}
\caption{Soundness error of polynomial commitment scheme. (* represents an imaginary linear code with relative distance 0.55)}
\label{table:benchmark-pc-3}
\end{table}

The theoretically computed soundness error for the setting used in the above benchmark experiment is large, even above 1, making it not usable in practice. The soundness error can be decreased by either increasing the number of tested tuples or by increasing the relative distance of the underlying linear code. However, the soundness error is not sensitive to the number of tested tuples and the length of the code is usually quite limited. Therefore, using a linear code with large relative distance is the only promising solution here. One of our conclusion would be high dimension polynomial commitment scheme is not worth using unless we can improve the relative distance of these linear code used in the constructions significantly. However, improving relative distance seems to be a difficult task.
