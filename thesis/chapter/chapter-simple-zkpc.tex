\chapter{Simple Zero-Knowledge Polynomial Commitments}

In this chapter, we describe a simple method to add zero-knowledge property to a given polynomial commitments scheme. This method uses random numbers to hide the actual coefficients and it works similar to one-time pad encryption.

\section{Protocol}

\textbf{Commitment Phase.}

Let $M_0 = u$ and $ M_0^{\prime} = \text{Enc}_1 \circ \text{Enc}_2 \circ \cdots \circ \text{Enc}_{t-1} (M_0) \in \mathbb{F}^{\underbrace{N \times N \times \cdots \times N}_{t-1 \text{ times}} \times m}$.
Let $PAD_0$ be a matrix with shape identical to $M_0^\prime$ filled with random elements from $\mathbb{F}$. Now let $M_0^{\prime\prime} = M_0^{\prime} \oplus PAD_0$, where $\oplus$ denotes elements-wise matrix addition.

Send $M_0^{\prime\prime}$ and $PAD_0$ to the verifier.

\textbf{Testing Phase.}

The testing phase consists of $t-1$ rounds, with each round reducing the number of dimensions by 1.

In round i, the verifier will send a random value $r_i \in \mathbb{F}^m$ to the prover.
The prover will compute a linear combination $M_i \in \mathbb{F}^{\underbrace{m \times m \times \cdots \times m}_{t-i \text{ times}}}$ across the last dimension of matrix $M_{i-1}$.
Namely, for $1 \le j_1,j_2, \cdots, j_{t-i} \le m$:
$$M_i[j_1,j_2, \cdots, j_{t-i}] = \sum_{k=1}^{m} r_{i}[k] \cdot M_{i-1}[j_1,j_2, \cdots, j_{t-i}, k]$$
Let 
$$
M_i^\prime = \text{Enc}_1 \circ \text{Enc}_2 \circ \cdots \circ \text{Enc}_{t - i - 1}(M_i)\in \mathbb{F}^{\underbrace{N \times N \times \cdots \times N}_{t-i-1 \text{ times}} \times m}
$$

The prover will also compute a linear combination $PAD_i \in \mathbb{F}^{\underbrace{N \times N \times \cdots \times N}_{t-i \text{ times}}}$ across the last dimension of matrix $PAD_{i-1}$.

Namely, for $1 \le j_1,j_2, \cdots, j_{t-i} \le N$:
$$PAD_i[j_1,j_2, \cdots, j_{t-i}] = \sum_{k=1}^{m} r_{i}[k] \cdot PAD_{i-1}[j_1,j_2, \cdots, j_{t-i}, k]$$
And let $PAD_i^\prime \in \mathbb{F}^{\underbrace{N \times N \times \cdots \times N}_{t-i-1 \text{ times}} \times m}$ be a sub-matrix truncated from $PAD_i$.

Let 
$$
M_i^{\prime\prime} = M_i^{\prime} \oplus PAD_i^\prime
$$
where $\oplus$ denotes element-wise addition.

Then the prover sends $M_i^{\prime\prime}$ and $PAD_i$ to the verifier.

Then the verifier will perform a probabilistic checking to make sure 
$M_0^\prime$, $M_1^\prime$, $M_2^\prime, \cdots, M_{t-1}^\prime$ are consistent with each other. Formally speaking, the verifier will sample $l$ random tuple $(i_1, i_2, \cdots, i_t)$ from space $\underbrace{[N] \times [N] \times \cdots \times [N]}_{t \text{ times}}$. Let $S_1$ denotes this set of sampled tuples.
For each tuple $(i_1, i_2, \cdots, i_t)$, the verifier will do the following checks:

\begin{itemize}
    \item 
    For every $i \in [t-1]$,
    $$
        \text{Enc}(M_i^\prime[i_1, i_2, \cdots, i_{t-i-1}, *])[i_{t-i}] \stackrel{?}{=} \sum_{k=1}^m r_i[k] \cdot M_{i-1}^{\prime}[i_1,i_2, \cdots, i_{t-i},k]
    $$
    Since the verifier do not have direct access to $M_0^\prime$, $M_1^\prime$, $M_2^\prime, \cdots, M_{t-1}^\prime$, a detour is necessary. $M_i^\prime[i_1, i_2, \cdots, i_{t-i-1}, *]$ in the left side of the equation can be computed as a element-wise subtraction as follows:
    $$
        M_i^{\prime\prime}[i_1, i_2, \cdots, i_{t-i-1}, *] - PAD_i^\prime[i_1, i_2, \cdots, i_{t-i-1}, *]
    $$
    And the right side of the equation is equivalent to 
    $$
        (\sum_{k=1}^m r_i[k] \cdot M_{i-1}^{\prime\prime}[i_1,i_2, \cdots, i_{t-i},k]) - PAD_i[i_1, i_2, \cdots, i_{t-i}]
    $$
    
    \item
    For every $2 \le i \le t-1$,
    $$
        PAD_i[i_1, i_2, \cdots, i_{t-i}] \stackrel{?}{=} \sum_{k=1}^m r_i[k] \cdot PAD_{i-1}[i_1,i_2, \cdots, i_{t-i},k]
    $$
\end{itemize}

Additionally, the verifier will sample another $p$ random distinct tuple $(i_1, i_2, \cdots, i_t)$ from space $\underbrace{[N] \times [N] \times \cdots \times [N]}_{t \text{ times}}$, with the restriction that 
$$(i_1, i_2, \cdots, i_t) \not\in S_1$$ 
Let $S_2$ denotes this set of sampled tuples. the verifier will check whether the following equation holds for each sampled tuple in $S_2$.
$$
    PAD_1[i_1, i_2, \cdots, i_{t-i}] \stackrel{?}{=} \sum_{k=1}^m r_1[k] \cdot PAD_{0}[i_1,i_2, \cdots, i_{t-i},k]
$$




\textbf{Evaluation Phase.}

Let $q_1, q_2, \cdots, q_t \in \mathbb{F}^{m}$ be vectors such that $g(x) =\langle q_1 \otimes q_2 \otimes \cdots \otimes q_t, u \rangle $. The evaluation phase is identical to the testing phase, except that in round $i$, the random value $r_i$ is replaced by $q_i$. If all consistent checks passed, then the verifier outputs $\langle M_{t-1}^{\prime}, q_t \rangle$ as $g(x)$, where $M_{t-1}^\prime$ can be computed as a element-wise subtraction between $M_{t-1}^{\prime\prime}$ and $PAD_{t-1}^\prime$.

\section{Formal Description}

\subsection{Notation}

\subsubsection{Fold Operation}

Define $\textbf{Fold}_i(X, r)$ to be the operation taking a linear combination of $X$ across the $i$-th dimension according to coefficient $r$. 

Namely, for indexes $j_1, \cdots, j_{i-1}, j_{i+1}, \cdots , j_{k} \ge 1$:
$$
\textbf{Fold}_i(X, r)[j_1, \cdots, j_{i-1}, j_{i+1}, \cdots , j_{k}] = \sum_{k=1}^{m} r_{i}[k] \cdot X[j_1, \cdots, j_{i-1}, k, j_{i+1}, \cdots , j_{k}]
$$

\subsubsection{Encode Operation}

Define $\textbf{Enc}_{1,\cdots,i}$ be short-hand for $\text{Enc}_1 \circ \text{Enc}_2 \circ \cdots \circ \text{Enc}_{i}$.

\subsection{Testing Phase (Proximity Test)}

In this section, we describe the testing phase in the above protocol formally in terms of a IOPP (interactive oracle proof of proximity) with point queries for the relation $R_\otimes(\mathbb{F}, C, m, N, t)$ between a prover $\textbf{P}$ and a verifier $\textbf{V}$.

The prover $\textbf{P}$ takes as input an instance $\mathbb{X} = (\mathbb{F}, C, m, N, t)$ and witness $\mathbb{W} = (M_0^{\prime\prime}, M_1^{\prime\prime}, \cdots, M_{t-1}^{\prime\prime}, PAD_1, PAD_2, \cdots, PAD_{t-1})$. The verifier $\textbf{V}$ takes as input the instance $\mathbb{X}$.

\begin{enumerate}
    \item \textit{Interactive phase}. 
    
    In the beginning, $\textbf{P}$ sends the proof message $M_0^{\prime\prime}$ and $PAD_0$ computed as:
$$
    M_0 = u \in \mathbb{F}^{m^t}
$$
$$
    M_0^{\prime} = \textbf{Enc}_{1,\cdots,t-1}(M_0) \in \mathbb{F}^{N^{t-1} \cdot m}
$$
$$
    M_0^{\prime\prime} = M_0^{\prime} \oplus PAD_0 \in \mathbb{F}^{N^{t-1} \cdot m}
$$
    Note that $PAD_0$ is a matrix with shape identical to $M_0^\prime$ filled with random elements from $\mathbb{F}$. And $\oplus$ denotes elements-wise matrix addition.
    
    For each round $i \in [t-1]$:
    \begin{itemize}
        \item $\textbf{V}$ sends random challenge message $r_i \in \mathbb{F}^m$.
        \item $\textbf{P}$ sends the proof message $M_i^{\prime\prime}$ and $PAD_i$ computed as:
$$
    PAD_i = \textbf{Fold}_{t-i+1}(PAD_{i-1}, r_i) \in \mathbb{F}^{N^{t-i}}
$$
$$
    M_i = \textbf{Fold}_{t-i+1}(M_{i-1}, r_i) \in \mathbb{F}^{m^{t-i}}
$$
$$
    M_i^\prime =  \textbf{Enc}_{1, \cdots, t- i - 1}(M_i) \in \mathbb{F}^{N^{t-i-1} \cdot m}
$$
$$
    M_i^{\prime\prime} = M_i^{\prime} \oplus PAD_i^\prime \in \mathbb{F}^{N^{t-i-1} \cdot m}
$$
    Note that $PAD_i^\prime$ is just a truncate version of $PAD_i$ such that $M_i^{\prime}$ and $ PAD_i^\prime$ will have the same shape. And $\oplus$ denotes elements-wise addition.

    \end{itemize}
    \item \textit{Query phase}. The verifier $\textbf{V}$ samples $l$ tuples of the form $(i_1, \cdots, i_t)$ in space $[N]^t$. Let $S_1$ denotes this set of sampled tuples.
    The verifier $\textbf{V}$ proceeds as follows for each tuple in $S_1$.
    
    For each $0 \le i \le t-1$, 
    the verifier $\textbf{V}$ will query $M_{i}^{\prime\prime}$ at $(i_1, \cdots, i_{t-i-1}, i_k)$ for each $i_k \in [m]$. 
    
    And for $1 \le i \le t-1$,
    the verifier $\textbf{V}$ will query $PAD_{i}$ at $(i_1, \cdots, i_{t-i-1}, i_k)$ for each $i_k \in [N]$.
    
    Then the verifier $\textbf{V}$ will check the following equation for $i \in [t-1]$:
$$
    \text{Enc}_{t-i}(M_i^\prime)[i_1, \cdots, i_{t-i}] \stackrel{?}{=} \textbf{Fold}_{t-i+1}(M_{i-1}^\prime, r_i) [i_1, \cdots, i_{t-i}]
$$
Since the verifier $\textbf{V}$ do not have direct access to $M_i^\prime$ and $M_{i-1}^\prime$, we check the following equivalent equation:
\begin{equation}
\label{eq:szkpctc_eq}
    \text{Enc}_{t-i}(M_i^{\prime\prime} \ominus PAD_{i}^\prime)[i_1, \cdots, i_{t-i}] 
    \stackrel{?}{=}
    \textbf{Fold}_{t-i+1}(M_{i-1}^{\prime\prime}, r_i) [i_1, \cdots, i_{t-i}] - PAD_i[i_1, \cdots, i_{t-i}]
\end{equation}
where $\ominus$ denotes element-wise subtraction.
Additionally, the verifier will also check the following equation for $2 \le i \le t-1$:
\begin{equation}
\label{eq:szkpctc_eq2}
    PAD_i[i_1, i_2, \cdots, i_{t-i}] 
    \stackrel{?}{=}
    \textbf{Fold}_{t-i+1}(PAD_{i-1}, r_i) [i_1, \cdots, i_{t-i}]
\end{equation}

Then the verifier will sample another $p$ random distinct tuple $(i_1, i_2, \cdots, i_t)$ from space $[N]^t$, with the restriction that 
$$(i_1, i_2, \cdots, i_t) \not\in S_1$$ 
Let $S_2$ denotes this set of sampled tuples. the verifier will check whether the following equation holds for each sampled tuple in $S_2$.
$$
    PAD_1[i_1, i_2, \cdots, i_{t-i}] 
    \stackrel{?}{=}
    \textbf{Fold}_{0}(PAD_{i-1}, r_i) [i_1, \cdots, i_{t-i}]
$$

\end{enumerate}

\subsection{Testing Phase Completeness}

\begin{lemma}
\label{lemma:szkpctcc}

IOPP = ($\textbf{P}$, $\textbf{V}$) has \textbf{perfect completeness}.

\end{lemma}
\begin{proof}
We begin by noting that the queries made by $\textbf{V}$ suffice to perform the checks in the query phase (see equation \ref{eq:szkpctc_eq} and equation \ref{eq:szkpctc_eq2}).

Next, observe that the verifier $\textbf{V}$ checks the following equation:
$$
    \text{Enc}_{t-i}(M_i^{\prime\prime} \ominus PAD_{i}^\prime) = \textbf{Fold}_{t-i+1}(M_{i-1}^{\prime\prime}, r_i) \ominus PAD_i 
$$
Note that the left side of this equation is equivalent to:
\begin{align}
\text{Enc}_{t-i}(M_i^{\prime\prime} \ominus PAD_{i}^\prime) &= \text{Enc}_{t-i}(M_i^\prime) \nonumber \\
&= \text{Enc}_{t-i}(\textbf{Enc}_{1, \cdots, t- i - 1}(M_i)) \nonumber \\
&= \textbf{Enc}_{1, \cdots, t-i}(M_i) \nonumber \\
&= \textbf{Enc}_{1, \cdots, t-i}(\textbf{Fold}_{t-i+1}(M_{i-1}, r_i)) \label{lb:exp1} \\
\end{align}
And the right side of this equation is equivalent to:
\begin{align}
\textbf{Fold}_{t-i+1}(M_{i-1}^{\prime\prime}, r_i) \ominus PAD_i &= \textbf{Fold}_{t-i+1}(M_{i-1}^\prime, r_i) \nonumber \\
&= \textbf{Fold}_{t-i+1}(\textbf{Enc}_{1, \cdots, t- i}(M_{i-1}), r_i) \label{lb:exp2} \\
\end{align}
Since both \textbf{Fold} and \textbf{Enc} operations are linear operation, expression \ref{lb:exp1} and  expression \ref{lb:exp2} are equivalent to each other. And equation \ref{eq:szkpctc_eq2} holds by definition.
The equations checked by the verifier $\textbf{V}$ holds.

\end{proof}




\subsection{Testing Phase Soundness}

\begin{lemma}
\label{lemma:szkpctc-soundness}


IOPP = ($\textbf{P}$, $\textbf{V}$) has soundness error at most:
$$
    \epsilon^\prime(\Delta_\otimes) = 
    \frac{ \epsilon(\Delta_\otimes)}{1 - (1 - (1 - \frac{1}{N^{t-1}})^l) \cdot (1 - \frac{1}{N^{t-1} - p})^p }
$$

\end{lemma}
\begin{proof}

We prove this lemma by arguing that the soundness (lemma \ref{lemma:pc-soundness}) of original proximity test implies the soundness of the simple zero-knowledge proximity test. And we will prove this argument by doing a reduction formally described in figure \ref{img:red-soundness}.

\begin{figure}[ht]
\centering
\resizebox{\textwidth}{!}{
\begin{bbrenv}[aboveskip=1cm, belowskip=1cm]{reduction-szk-soundness-1}
\begin{bbrbox}[name=\textbf{P}, minheight=6cm]
% The Box’s content

    \begin{bbrenv}[aboveskip=1cm, belowskip=1cm]{reduction-szk-soundness-2}
    \begin{bbrbox}[name=$\textbf{P}^\prime$, minheight=4cm, namepos=middle]
    \end{bbrbox}
    \bbrinput{$M_0^\prime$}
    
    
    \bbrqryto{top={$M_0^{\prime\prime}$, $PAD_0$}, length=4cm}
    \bbrqryfromto{top=$r_i$, length=4cm}{bottom={$M_i^{\prime\prime}$, $PAD_i$}, length=4cm}
    \bbrqryspace{0.5cm}
    \bbrqryfromto{top={Query $M_i^{\prime\prime}[I]$, $PAD_i[I]$}, length=4cm}{bottom={$M_i^{\prime\prime}[I]$, $PAD_i[I]$}, length=4cm}
    
    \end{bbrenv}


\end{bbrbox}
% Commands to display communication, input output etc
\bbrinput{$M_0^\prime$}

\begin{bbroracle}{reduction-szk-soundness-3}[hdistance=4.5cm, vdistance=\baselineskip]

\begin{bbrbox}[name=$\textbf{V}$, minheight=6cm, namepos=middle]
\end{bbrbox}
\bbroutput{Pass / Fail}
\end{bbroracle}

\bbroracleqryto{top={$M_0^{\prime}$}}
\bbroracleqryfromto{top=$r_i$}{bottom={$M_i^{\prime} = M_i^{\prime\prime} \ominus PAD_i^\prime$}}
\bbroracleqryspace{0.5cm}
\bbroracleqryfromto{top=Query $M_0^\prime[I]$}{bottom={$M_0^\prime[I]$}}
\bbroracleqryspace{0.5cm}
\bbroracleqryfromto{top=Query $M_i^\prime[I]$}{bottom={$M_i^\prime[I] = M_i^{\prime\prime}[I] - PAD_i[I] $}}

\end{bbrenv}
}
\caption{Reduction from Simple ZK Proximity Test to Original Proximity Test}
\label{img:red-soundness}
\end{figure}

Let $E_1$ be the event that the following equation holds for all sampled tuples,
$$
    \textbf{Fold}_{t}(PAD_{0}^\prime, r_0) = PAD_1
$$
For the simple zero-knowledge proximity test, suppose we have an malicious prover algorithm $\textbf{P}^\prime$ that can pass the proximity test with probability $p_0$, we will run it as a black box and construct another malicious prover algorithm $\textbf{P}$ that can pass the original proximity test with probability $p_0$ if event $E_1$ happens.

\begin{itemize}
    \item $\textbf{P}$ will receive an input \mzeroprime representing the tensor code matrix. \textbf{P} pass \mzeroprime to $\textbf{P}^\prime$.
    
    \item $\textbf{P}^\prime$ sends out $M_0^{\prime\prime}$ as the commitment to the underlying tensor code matrix. 
    \textbf{P} sends $M_0^\prime$ to the verifier \textbf{V}  as the commitment to the underlying tensor code matrix.
    
    \item For $i \in [t-1]$, verifier \textbf{V} will send the random linear combination coefficients $r_i$ to \textbf{P}. \textbf{P} simply forward $r_i$ to $\textbf{P}^\prime$. $\textbf{P}^\prime$ will output $M_i^\prime$ and $PAD_i$. \textbf{P} then compute $M_i^\prime$ as element-wise subtraction of $M_i^\prime$ and $PAD_i$ and then send $M_i^\prime$ to the verifier \textbf{V}. 
    
    Note that $PAD_i^\prime$ is the truncated version of $PAD_i$.
    \item If verifier \textbf{V} query $M_0^\prime$ at index $I$, \textbf{P} return $M_0^\prime[I]$ directly.
    
    \item If verifier \textbf{V} query $M_i^\prime$ at index $I$ for $i \in [t-1]$, \textbf{P} query $\textbf{P}^\prime$ for $M_i^{\prime\prime}$ and $PAD_i$ at index $I$. \textbf{P} will return $M_i^\prime$ computed as $M_i^{\prime\prime}[I] - PAD_i[I]$.
\end{itemize}

During the query phase, the original verifier \textbf{V} will check the following equation:
$$
    \text{Enc}_{t-i}(M_i^\prime) = \textbf{Fold}_{t-i+1}(M_{i-1}^\prime, r_i)
$$
And our construction \textbf{P} will act like the zero-knowledge verifier $\textbf{V}^\prime$ and check the following equation:
$$
    \text{Enc}_{t-i}(M_i^{\prime\prime} \ominus PAD_{i}^\prime) = \textbf{Fold}_{t-i+1}(M_{i-1}^{\prime\prime}, r_i) \ominus PAD_i
$$
Looking at the left side of these two equation, $\text{Enc}_{t-i}(M_i^\prime)$ is identical to $\text{Enc}_{t-i}(M_i^{\prime\prime} \ominus PAD_{i}^\prime)$, since $M_i^\prime = M_i^{\prime\prime} \ominus PAD_{i}^\prime$ by construction.

And since equation \ref{eq:szkpctc_eq2} is checked in the simple zero-knowledge proximity test and event $E_1$ happens, the right side of these two equations are also identical to each other.
\begin{align}
\textbf{Fold}_{t-i+1}(M_{i-1}^\prime, r_i) 
&= \textbf{Fold}_{t-i+1}(M_{i-1}^{\prime\prime} \ominus PAD_{i-1}^\prime, r_i) \nonumber \\
&= \textbf{Fold}_{t-i+1}(M_{i-1}^{\prime\prime}, r_i) \ominus \textbf{Fold}_{t-i+1}(PAD_{i-1}^\prime, r_i) \nonumber \\
&= \textbf{Fold}_{t-i+1}(M_{i-1}^{\prime\prime}, r_i) \ominus PAD_i \nonumber \\
\end{align}

Therefore, as long as malicious prover $\textbf{P}^\prime$ can pass the check in zero-knowledge proximity test, our constructed prover \textbf{P} can pass the check in original proximity test if event $E_1$ happens.

The probability event $E_1$ happens could be computed as follows:

$$
    1 - (1 - (1 - \frac{1}{N^{t-1}})^l) \cdot (1 - \frac{1}{N^{t-1} - p})^p 
$$

Therefore, the soundness error of the simple zero-knowledge protocol is at most

$$
    \epsilon^\prime(\Delta_\otimes) = 
    \frac{ \epsilon(\Delta_\otimes)}{1 - (1 - (1 - \frac{1}{N^{t-1}})^l) \cdot (1 - \frac{1}{N^{t-1} - p})^p }
$$

\end{proof}


\subsection{Testing Phase Zero-Knowledge}

\begin{definition}

A interactive oracle proof of proximity IOPP = ($\textbf{P}$, $\textbf{V}$) for a relation $R$ is \textbf{perfect zero-knowledge} if there exists a polynomial-time simulator algorithm $\textbf{S}$ such that, for every $(\mathbb{X}, \mathbb{W}) \in R$ and choice of verifier randomness $\rho$, the random variables $\textbf{S}^{\textbf{V}(\mathbb{X};\rho)}(\mathbb{X})$ and $\text{View}(\textbf{P}(\mathbb{X}, \mathbb{W}), \textbf{V}(\mathbb{X};\rho))$ are identically distributed.
 
\end{definition}

\begin{lemma}
\label{lemma:szkpc-zk}

IOPP = ($\textbf{P}$, $\textbf{V}$) is \textbf{perfect zero-knowledge}

\end{lemma}
\begin{proof}

For every $(\mathbb{X}, \mathbb{W}) \in R_\otimes$ and choice of verifier randomness $\rho$, we can construct the polynomial-time simulator algorithm $\textbf{S}$ as follows:

\begin{itemize}
    \item Query $M_0^{\prime\prime}$ or internally access $PAD_0$ at index $I$:
    
    If index I has never been accessed before, generate a random element $x$ from field $\mathbb{F}$. Store this value $x$ in a internal dictionary and return it.
    
    Otherwise lookup index $I$ in the internal dictionary and return the stored value.
    
    \item Query $M_i^{\prime\prime}$ and $PAD_i$ at index $I$ for $i \in [t-1]$:

    If index I has never been accessed before, recursively access the corresponding values in $M_{i-1}^{\prime\prime}$ and $PAD_{i-1}$ and take the linear combination of them as $x$.
    Store this value $x$ in a internal dictionary and return it.
    
    Otherwise lookup index $I$ in the internal dictionary and return the stored value.    
    
\end{itemize}

Both the random variables $\textbf{S}^{\textbf{V}(\mathbb{X};\rho)}(\mathbb{X})$ and $\text{View}(\textbf{P}(\mathbb{X}, \mathbb{W}), \textbf{V}(\mathbb{X};\rho))$ are uniformly distributed. They are identically distributed.

\end{proof}