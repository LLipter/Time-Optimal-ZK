\chapter{Preliminaries}

\section{Combinatorics}

\begin{definition}[$d$-regular Graph]

A graph $G = (V, E)$ is $d$-regular if every vertex in $V$ has degree $d$.

\end{definition}

\begin{definition}[Set]

$[n]$ is the shorthand for the set $\{i: 1 \le i \le n\}$

\end{definition}

\section{Interactive Oracle Proofs}

\begin{definition}[Relation]

A \textbf{relation} $R$ is a set of pairs $(\mathbb{X}, \mathbb{W})$ where $\mathbb{X}$ is the instance and $\mathbb{W}$ is the witness. The corresponding \textbf{language} $L(R)$ is the set of instances $\mathbb{X}$ for which there exists a witness $\mathbb{W}$ such that $(\mathbb{X}, \mathbb{W}) \in R$.

\end{definition}

\begin{definition}[Interactive Proof (IP)]
An \textbf{Interactive Proof (IP)} is defined by a pair of interactive randomized
algorithms \textbf{IP = (P, V)}, where \textbf{P} denotes the prover and \textbf{V} the verifier. 
The number of rounds of interaction is called the \textbf{round complexity} of the system. 
During a single round, the prover sends a message to the verifier, and the verifier reply a message back to the prover. 
The \textbf{proof length} is the sum of lengths of all messages sent by the prover.
We denote by $\langle P \leftrightarrow V \rangle (\mathbb{X}, \mathbb{W})$ the output of \textbf{V} after interacting with \textbf{P} on instance $\mathbb{X}$ and witness $\mathbb{W}$; this output is either \textsc{accept} or \textsc{reject}.

An interactive proof IP = (P, V) for a relation R has completeness 1 and soundness error $\epsilon$ if the following holds.

\begin{enumerate}
    \item \textbf{Completeness.}
    For every pair $(\mathbb{X}, \mathbb{W}) \in R$, the probability that $P(\mathbb{X}, \mathbb{W})$ convinces $V(\mathbb{X})$ to accept is 1.
    
    \item \textbf{Soundness.}
    For every instance $\mathbb{X} \not\in L(R)$ and malicious prover $\tilde{P}$, the probability that $\tilde{P}$ convinces $V(\mathbb{X})$ to accept is at most $\epsilon$.
\end{enumerate}
\end{definition}

\begin{definition}[Point Query Interactive Oracle Proof (IOP)]
An \textbf{Interactive Oracle Proof (IOP)} is defined by a pair of interactive randomized
algorithms \textbf{IOP = (P, V)}, where \textbf{P} denotes the prover and \textbf{V} the verifier. 
The number of rounds of interaction is called the \textbf{round complexity} of the system. 
During a single round the prover sends a message to which the verifier is given oracle access, and the verifier responds with a message to the prover. 
The \textbf{proof length} is the sum of lengths of all messages sent by the prover. Specifically, the prover is allowed to send a large array message $\pi$ to the verifier, and the verifier is allowed to query this message $\pi$ at position $i \in [N]^t$. The message $\pi$ will work as an oracle and the verifier will learn $\pi[i]$ through this query.
The \textbf{query complexity} of the protocol is the number of entries read by \textbf{V} from the various prover messages.
We denote by $\langle P \leftrightarrow V \rangle (\mathbb{X}, \mathbb{W})$ the output of \textbf{V} after interacting with \textbf{P} on instance $\mathbb{X}$ and witness $\mathbb{W}$; this output is either \textsc{accept} or \textsc{reject}.

An interactive oracle proof IOP = (P, V) for a relation R has completeness 1 and soundness error $\epsilon$ if the following holds.

\begin{enumerate}
    \item \textbf{Completeness.}
    For every pair $(\mathbb{X}, \mathbb{W}) \in R$, the probability that $P(\mathbb{X}, \mathbb{W})$ convinces $V(\mathbb{X})$ to accept is 1.
    
    \item \textbf{Soundness.}
    For every instance $\mathbb{X} \not\in L(R)$ and malicious prover $\tilde{P}$, the probability that $\tilde{P}$ convinces $V(\mathbb{X})$ to accept is at most $\epsilon$.
\end{enumerate}
\end{definition}

\begin{definition}[Interactive Oracle Proof of Proximity (IOPP)]
An \textbf{Interactive Oracle Proof of Proximity (IOPP)} is defined by a pair of interactive randomized
algorithms \textbf{IOPP = (P, V)}, where \textbf{P} denotes the prover and \textbf{V} the verifier. 
The number of rounds of interaction is called the \textbf{round complexity} of the system. 
During a single round the prover sends a message to which the verifier is given oracle access, and the verifier responds with a message to the prover. 
The \textbf{proof length} is the sum of lengths of all messages sent by the prover. Specifically, the prover is allowed to send a large array message $\pi$ to the verifier, and the verifier is allowed to query this message $\pi$ at position $I$. The message $\pi$ will work as an oracle and the verifier will learn $\pi[I]$ through this query.
The \textbf{query complexity} of the protocol is the number of entries read by \textbf{V} from the various prover message.
We denote by $\langle P \leftrightarrow V \rangle (\mathbb{X}, \mathbb{W})$ the output of \textbf{V} after interacting with \textbf{P} on instance $\mathbb{X}$ and witness $\mathbb{W}$; this output is either \textsc{accept} or \textsc{reject}.
The protocol's goal is to show that a particular string is close to a valid witness.
An interactive oracle proof of proximity IOPP = (P, V) for a relation R has completeness 1 and soundness error $\epsilon$ with distance function $\Delta(w_1, w_2) \in \mathbb{F}$ ($w_1, w_2 \in \mathbb{F}^N$) if the following holds.

\begin{enumerate}
    \item \textbf{Completeness.}
    For every pair $(\mathbb{X}, \mathbb{W}) \in R$, the probability that $P(\mathbb{X}, \mathbb{W})$ convinces $V^{\mathbb{W}}(\mathbb{X})$ to accept is 1.
    
    \item \textbf{Soundness.}
    For every instance $\mathbb{X} \not\in L(R)$ and malicious prover $\tilde{P}$, the probability that $\tilde{P}$ convince $V^{\mathbb{W}}(\mathbb{X})$ to accept is at most $\epsilon(\Delta(\mathbb{W}, R|_{\mathbb{X}}))$. Here the soundness error $\epsilon$ is a function of the $\Delta$-distance of $\mathbb{W}$ to the set of valid witnesses $R|_{\mathbb{X}} := \{ \mathbb{W}^\prime | (\mathbb{X}, \mathbb{W}^\prime) \in R \}$.
\end{enumerate}
\end{definition}

In practice, we use Merkle tree commitment to compile the IOP or IOPP to an real argument system. Each element in the large array message $\pi$ sent by the prover will be considered to be a leaf node of a Merkle tree. And the corresponding Merkle tree root will be sent to the verifier instead. For each query at position $I$, the prover will responds with $\pi[I]$ and the corresponding Merkle tree path, which will be authenticated later by the verifier.

\begin{definition}

A interactive oracle proof IOP = ($\textbf{P}$, $\textbf{V}$) for a relation $R$ is \textbf{semi-honest verifier zero-knowledge} if there exists a polynomial-time simulator algorithm $\textbf{S}$ such that, for every $(\mathbb{X}, \mathbb{W}) \in R$ and choice of verifier randomness $\rho$, the random variables $\textbf{S}^{\textbf{V}(\mathbb{X};\rho)}(\mathbb{X})$ and $\text{View}(\textbf{P}(\mathbb{X}, \mathbb{W}), \textbf{V}(\mathbb{X};\rho))$ are identically distributed.
 
\end{definition}

\begin{definition}

A interactive oracle proof of proximity IOPP = ($\textbf{P}$, $\textbf{V}$) for a relation $R$ is \textbf{semi-honest verifier zero-knowledge} if there exists a polynomial-time simulator algorithm $\textbf{S}$ such that, for every $(\mathbb{X}, \mathbb{W}) \in R$ and choice of verifier randomness $\rho$, the random variables $\textbf{S}^{\textbf{V}(\mathbb{X};\rho)}(\mathbb{X})$ and $\text{View}(\textbf{P}(\mathbb{X}, \mathbb{W}), \textbf{V}(\mathbb{X};\rho))$ are identically distributed.
 
\end{definition}

\begin{definition}

Let $A$ be an algorithm with adaptive query access to oracles $O_1, \cdots, O_n$. Let $Q$ be a stateful query-checker algorithm which receives the adaptive queries of $A$ and may output $\perp$ at any point. We say that $A$ is a $Q$-query algorithm if $Q$ never outputs $\perp$.
 
\end{definition}

\begin{definition}

A interactive oracle proof of proximity IOPP = ($\textbf{P}$, $\textbf{V}$) for a relation $R$ is \textbf{perfect zero-knowledge} with query-checker $Q$ if there exists a polynomial-time simulator algorithm $\textbf{S}$ such that, for every $(\mathbb{X}, \mathbb{W}) \in R$, 
$Q$-query algorithm $\widetilde{V}$ and the choice of verifier randomness $\rho$, the random variables $\textbf{S}^{\widetilde{V}(\mathbb{X};\rho)}(\mathbb{X})$ and $\text{View}(\textbf{P}(\mathbb{X}, \mathbb{W}), \widetilde{V}(\mathbb{X};\rho))$ are identically distributed.
 
\end{definition}

\section{Polynomials}

\begin{definition}[Monomials of Polynomial]
A polynomial g over $\mathbb{F}$ is an expression consisting of a sum of \textbf{monomials} where each monomial is the product of a constant (from $\mathbb{F}$) and powers of one or more variables (which take values from $\mathbb{F}$); all arithmetic is performed over $\mathbb{F}$.
\end{definition}

\begin{definition}[Degree of Polynomial]
The degree of a monomial is the sum of the exponents of variables in the monomial; the (total) degree of a polynomial g is the maximum degree of any monomial in g. Furthermore, the degree of a polynomial g in a particular variable $x_i$ is the maximum exponent that $x_i$ takes in any of the monomials in g.
\end{definition}

\begin{definition}[Multivariate / Univariate Polynomial]
A \textbf{multivariate} polynomial is a polynomial with more than one variable; otherwise it is called a \textbf{univariate} polynomial.
\end{definition}

\begin{definition}[Multilinear Polynomial]
A \textbf{multivariate} polynomial is called a \textbf{multilinear} polynomial if the degree of the polynomial in each variable is at most one.
\end{definition}

\begin{definition}[Polynomial Commitment]
A \textbf{Polynomial Commitment} is an interactive proof (IP), which defines a relation $R = ((C, x, y), (\phi(\cdot))$. 
And it consists of three algorithms (\textsc{PC.Setup}, \textsc{PC.Commit}, \textsc{PC.Verify}) and an evaluation protocol \textsc{PC.Eval}, where:

\begin{itemize}
    \item \textsc{PC.Setup}$(\lambda, d)$: the algorithm outputs public parameters $pp$ for committing to polynomials of degree $d$. The parameters $pp$ include a specification of an field $\mathbb{F}$.
    
    \item \textsc{PC.Commit}$(pp, \phi(\cdot))$: the algorithm outputs a commitment $\mathcal{C}$ of the polynomial $\phi(\cdot)$ with degree at most $d$.
    
    \item \textsc{PC.Verify}$(pp, \phi(\cdot), \mathcal{C})$: given $\phi(\cdot), \mathcal{C}$, the algorithm checks if $\mathcal{C}$ is a valid commitment for polynomial $\phi(\cdot)$  with degree at most $d$. The algorithm outputs \textsc{accept} or \textsc{reject}.
    
    \item \textsc{PC.Eval}$(\mathcal{P}(\phi(\cdot)), \mathcal{V}(pp, \mathcal{C}, x, y))$ : this is an  interactive protocol between a prover $\mathcal{P}$ who has the polynomial $\phi(\cdot)$ as private input and a verifier $\mathcal{V}$ who has $(\mathcal{R}, x, y)$ as common public input. The purpose of the protocol is to convince the verifier that $\phi(x) = y$ and the degree of $\phi(\cdot)$ is at most $d$.
\end{itemize}

\end{definition}

\textbf{Completeness.} Let $\mathcal{C} \leftarrow \textsc{PC.Commit}(pp, \phi(\cdot))$ be the commitment of a polynomial. 
For all polynomials $\phi(\cdot)$ and all points $x$, with probability 1 the verification \textsc{PC.Verify}$(pp, \phi(\cdot), \mathcal{C})$ outputs \textsc{accept}. And likewise, $\mathcal{V}$ output \textsc{accept} in interaction with $\mathcal{P}$ in the \textsc{PC.Eval} protocol on valid inputs. The formal completeness requirement is:
$$
Pr
\begin{pmatrix}
 b_1 = \textsc{accept} \wedge b_2 = \textsc{accept}: \\
 pp \leftarrow \textsc{PC.Setup}(\lambda, d) \\
 \mathcal{C} \leftarrow \textsc{PC.Commit}(pp, \phi(\cdot)) \\
 b_1 \leftarrow \textsc{PC.Verify}(pp, \phi(\cdot), \mathcal{C}) \\
 (\perp, b_2)\leftarrow \textsc{PC.Eval}(\mathcal{P}(\phi(\cdot)), \mathcal{V}(pp, \mathcal{C}, x, y))
\end{pmatrix}
= 1
$$


\textbf{Binding.} For all adversaries $\mathcal{A}$, the binding error $\epsilon_{bind}$ is defined to be the following probability:
$$
\epsilon_{bind} = Pr
\begin{pmatrix}
 b_1 = \textsc{accept } \wedge \\
 b_2 = \textsc{accept } \wedge \\
 \phi(\cdot) \neq \phi^\prime(\cdot): \\
 pp \leftarrow \textsc{PC.Setup}(\lambda, d) \\
 (\mathcal{C}, \phi(\cdot), \phi^\prime(\cdot)) \leftarrow \mathcal{A}(x) \\
 b_1 \leftarrow \textsc{PC.Verify}(pp, \phi(\cdot), \mathcal{C}) \\
 b_2 \leftarrow \textsc{PC.Verify}(pp, \phi^\prime(\cdot), \mathcal{C})
\end{pmatrix}
$$



\textbf{Hiding.} The polynomial commitment scheme is hiding if the commitments to distinct polynomials are statistically indistinguishable. Formally speaking, for all adversaries $\mathcal{A} = (\mathcal{A}_0, \mathcal{A}_1)$, the hiding error $\epsilon_{hide}$ is defined to be the following probability:

$$
\epsilon_{hide} = Pr
\begin{pmatrix}
 b = b^\prime : \\
 pp \leftarrow \textsc{PC.Setup}(\lambda, d) \\
 (\phi_0(\cdot), \phi_1(\cdot)) \leftarrow \mathcal{A}_0 \\
 b \overset{{\scriptscriptstyle\$}}{\leftarrow} \{0, 1\} \\
 \mathcal{C} \leftarrow \textsc{PC.Commit}(pp, \phi_b(\cdot)) \\
 b^\prime \leftarrow \mathcal{A}_1(\mathcal{C})
\end{pmatrix}
$$


\textbf{Soundness.} For all malicious prover $\tilde{P}$, the soundness error $\epsilon_{sound}$ is defined to be the following probability:
$$
\epsilon_{sound} = Pr
\begin{pmatrix}
 b = \textsc{accept } \wedge \\
 pp \leftarrow \textsc{PC.Setup}(\lambda, d) \\
 \forall \tilde{\mathcal{P}} \\
 \mathcal{C} \leftarrow \textsc{PC.Commit}(pp, \phi(\cdot)) \\
 (\perp, b)\leftarrow \textsc{PC.Eval}(\tilde{\mathcal{P}}(\phi(\cdot)), \mathcal{V}(pp, \mathcal{C}, x, y))
\end{pmatrix}
$$



\textbf{Zero-knowledge.} At the end of the protocol, the verifier will know the evaluation result of the polynomial at some evaluation points, but nothing more than that. The polynomial commitment scheme is zero-knowledge if the view of the verifier $\mathcal{V}$ generated by the interactive protocol \textsc{PC.Eval} is statistically indistinguishable to the view of the verifier generated by a simulator $\mathcal{S}$. Formally speaking, for all adversaries, the zero-knowledge error $\epsilon_{zk}$ is defined to be the following probability:
$$
\epsilon_{zk} = Pr
\begin{pmatrix}
 b = b^\prime : \\
 pp \leftarrow \textsc{PC.Setup}(\lambda, d) \\
 b \overset{{\scriptscriptstyle\$}}{\leftarrow} \{0, 1\} \\
 \mathcal{C} \leftarrow \textsc{PC.Commit}(pp, \phi(\cdot)) \\
 t_0 \leftarrow \mathcal{S} \\
 t_1 \leftarrow \textsc{View}(\textsc{PC.Eval}(\mathcal{P}(\phi(\cdot)), \mathcal{V}(pp, \mathcal{C}, x, y))) \\
 b^\prime \leftarrow \mathcal{A}(t_b, C)
\end{pmatrix}
$$

\section{Linear Codes}

\begin{definition}[Linear Code]
If $\mathbb{F}$ is a field and $C \subset \mathbb{F}^n$ is a subspace of $\mathbb{F}^n$ then C is said to be a linear code.
\end{definition}

The weight of a codeword is the number of its elements that are nonzero and the distance between two codewords is the \textbf{Hamming distance} between them, that is, the number of elements in which they differ. The distance $d$ of the linear code is the minimum weight of its nonzero codewords, or equivalently, the \textbf{minimum distance} between distinct codewords. A linear code of length $n$, dimension $k$, and distance $d$ is called an $[n,k,d]$ code.

As $C$ is a subspace, there exists a basis $c_1, c_2, \cdots, c_k$ where $k$ is the dimension of the subspace. Any codeword can be expressed as the linear combination of these basis vectors. We can write these vectors in matrix form as the rows of a $k \times n$ matrix. Such a matrix is called a \textbf{generator matrix}.

Additionally, any linear combination of valid codeword is also a valid codeword, which is the \textbf{linearity property}. Formally speaking, given $x_1, x_2, \cdots, x_n$ are valid codeword, and given $r_1, r_2, \cdots, r_n$ are some constants, $x^\prime = r_1 x_1 + r_2 x_2 + \cdots + r_n x_n$ is also a valid codeword.

\begin{definition}[Tensor Product Code]
The tensor product code $C^{\otimes t}$ is the linear code in $\mathbb{F}^{n^t}$ with message length $k^t$, block length $n^t$ and distance $d^t$ where any axis-parallel line of elements is in C.
\end{definition}

\begin{definition}[Relation $R_\otimes$]
\label{def:relation-prox}
The relation $R_\otimes$ is the sets of tuples
$$
    (\mathbb{X}, \mathbb{W}) = ((\mathbb{F}, C, l, q, t), (c_0^{0}, \{c_1^{(s)}\}_s, \cdots, \{c_{t-1}^{(s)}\}_s))
$$ 
such that $c_0^{0} \in (C^{\otimes t})^l$ and for all $r \in [t-1]$ and $s \in [q]$, we have $c_r^{(s)} \in (C^{\otimes t-r})^k$.
\end{definition}

\begin{definition}[Relation $R_{cons}$]
\label{def:relation-cons}
The relation $R_{cons}$ is the sets of tuples
$$
    (\mathbb{X}, \mathbb{W}) = ((\mathbb{F}, C, l, q, t, \{\texttt{q}^{(s)}\}), c)
$$ 
such that $c = \textsc{Enc}_{C^{\otimes t}}(f) \in \mathbb{F}^{l\cdot n^t}$ for some $f \in \mathbb{F}^{l\cdot k^t}$, 
for each $s \in [q]$, $\texttt{q}^{(s)} = (\texttt{q}_0^{(s)}, \cdots, \texttt{q}_t^{(s)}) \in \mathbb{F}^{l} \times (\mathbb{F}^k)^t$, 
and for all $s \in [q]$, $\langle \otimes_{i}\texttt{q}_i^{(s)} , f \rangle = v^{(s)}$.

\end{definition}

\begin{definition}[Distance $\Delta_\otimes$]
Let $\mathbb{W} = (c_0^{0}, \{c_1^{(s)}\}_s, \cdots, \{c_{t-1}^{(s)}\}_s)$ be such that $c_0^{(0)} \in \mathbb{F}^{l \cdot n^t}$ and, for all $r \in [t-1]$ and $s \in [q]$, we have $c_r^{(s)} \in \mathbb{F}^{k \cdot n^{t-r}}$. Given $\mathbb{X} = (\mathbb{F}, C, l, q, t)$, the $\Delta_\otimes$ distance of $\mathbb{W}$ to $R_\otimes|_{\mathbb{X}}$ is
$$
    \Delta_\otimes(\mathbb{W}, R_\otimes|_{\mathbb{X}}) := \max \{\Delta_0, \Delta_1, \cdots, \Delta_{t-1}\}
$$
where $\Delta_0 := \Delta(c_0^{(0)}, C^{\otimes t})$ and $\forall r \in [t-1], \Delta_r := \Delta(\{c_r^{(s)}\}_s, C^{\otimes t-r})$.
\end{definition}