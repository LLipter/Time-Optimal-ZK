\chapter{Zero-Knowledge Linear Code}

% http://www.cs.cmu.edu/~odonnell/toolkit13/lecture12.pdf
\section{Random d-regular Bipartite Graph}
\label{sec:randomgraph}
To make sure each vertex has degree $d$, we can first sample $d$ random perfect matching for 2 sets of $n$ vertices. Then take the union of them. Note that it's possible to generate parallel edges. But this should not be a concern for our purpose here. And it can be shown that this happens with low probability.

\RestyleAlgo{ruled}
%% This is needed if you want to add comments in
%% your algorithm with \Comment
\SetKwComment{Comment}{/* }{ */}
\begin{algorithm}[hbt!]
\caption{Random d-regular Bipartite Graph Generation}
\label{alg:randomgarph}
\KwData{$n \geq 0$, $d <= n$}
\KwResult{A random $d$-regular bipartite graph $G=(L, R, E)$ with $|L| = |R| = n$}

$L \gets \text{a set of } n \text{ nodes}$\;
$R \gets \text{a set of } n \text{ nodes}$\;
$E \gets \emptyset$\;
$P \gets [1, 2, \cdots , n]$\;
\For{$i$ in $1, 2, \cdots , d$}{
    Permute $P$ randomly \Comment*[r]{sample a perfect matching}
    \For{$j$ in $1, 2, \cdots , n$}{
        $E \gets E \cup (L_j, R_{P_j})$ \;
    }
}
\Return{(L, R, E)}
\end{algorithm}


\section{Expander Graph}

\begin{lemma}
\label{lemma:randomgraph}

For any $0 < \epsilon < 1$, there exist a degree $d$ such that a random $d$-regular bipartite graph $G=(L, R, E)$ with $|L| = |R| = n$ generated according to algorithm \ref{alg:randomgarph} satisfy the following property with high probability.

    \begin{itemize}
        \item Expansion: For every set $X \subset L$ with $|X| \ge \epsilon n$, if $Y$ is the set of neighbors of $X$ in $G$, then $|Y| \ge (1 - \epsilon)  n$.
    \end{itemize}

\end{lemma}

\begin{proof}

Negating the statement, we can say that the randomly generated graph $G$ doesn't satisfy the expansion property if and only if $\exists S \subseteq L$, $|S| \ge \epsilon n$, $\exists M \subseteq R$, $|M| \ge  \epsilon n$ such that there's no edge connecting between set $S$ and set $M$. We bound the probability that this negating statement is true as follows:

For every vertex $a \in L$ and every vertex $b \in R$, the probability that $a$ and $b$ are not connected in the random graph $G$ is:

$$P_1 = (\frac{n-1}{n})^{d}$$

For a set of vertices $S \subset L$ with $|S| = s \ge \epsilon n$, the probability that non of vertices in $S$ is connected to $b$ is:

$$P_2 = (P_1)^s = (\frac{n-1}{n})^{d s}$$

The probability that there exists at least $\epsilon n$ vertices in $R$ are not connected to any vertex in $S$ is:

$$P_3 = \binom{n}{\epsilon n} (P_2)^{\epsilon n} = \binom{n}{\epsilon n} (\frac{n-1}{n})^{d s \epsilon n}$$

For $0 \le x \le 1$, we denote the binary entropy function to be:

$$H(x) = -x\log_2 x - (1-x)\log_x (1-x)$$ 
where we adopt the convention that $0 \log_2 0 = 0$.

Then, we take a union bound over all possible sets $S$, 

\begin{align}
% bound 5
P_4 &= \sum_{s=\epsilon n}^{n} \binom{n}{s} P_3 \nonumber \\
    &= \sum_{s=\epsilon n}^{n} \binom{n}{s} \binom{n}{\epsilon n} (\frac{n-1}{n})^{d s \epsilon n} \nonumber \\
    &\le \sum_{s=\epsilon n}^{n} \binom{n}{s} \binom{n}{\epsilon n} (\frac{n-1}{n})^{d \epsilon^2 n^2} 
    && \text{since } s \ge \epsilon n \text{ and } \frac{n-1}{n} < 1 \nonumber \\
    &\le \sum_{s=\epsilon n}^{n} \binom{n}{s} 2^{n H(\frac{\epsilon n}{n})} (\frac{n-1}{n})^{d \epsilon^2 n^2} 
    && \binom{n}{k} \le 2^{n H(\frac{k}{n})} \nonumber \\
    &= \sum_{s=\epsilon n}^{n} \binom{n}{s} 2^{n H(\epsilon)} ((1 - \frac{1}{n})^{n})^{d \epsilon^2 n} \nonumber \\
    &\le \sum_{s=\epsilon n}^{n} \binom{n}{s} 2^{n H(\epsilon)} (\frac{1}{e})^{d \epsilon^2 n} 
    && (1 - \frac{1}{x})^x \le \frac{1}{e} \text{ for } x \ge 1 \text{ (lemma \ref{lemma:(1-1x)x})} \nonumber \\
    &= \sum_{s=\epsilon n}^{n} \binom{n}{s} (e^{ H(\epsilon) \ln 2  - d \epsilon^2})^n \nonumber \\
    &\le \sum_{s=0}^{n} \binom{n}{s} (e^{ H(\epsilon) \ln 2 - d \epsilon^2})^n \nonumber \\
    &= 2^n (e^{ H(\epsilon) \ln 2 - d \epsilon^2})^n 
    && \sum_{i=0}^n \binom{n}{i} = 2^n \nonumber \\
    &= (e^{\ln 2 + H(\epsilon) \ln 2 - d \epsilon^2})^n \nonumber \\
\end{align}

$P_4$ is the probability that a randomly generated graph $G$ doesn't satisfy the expansion property. Suppose we want the failing probability be smaller than $p$, let $(e^{\ln 2 + H(\epsilon) \ln 2 - d \epsilon^2})^n < p$.
By rearranging the above equation, we have $ d > \frac{\ln 2 + H(\epsilon) \ln 2 - \frac{\ln p}{n}}{\epsilon^2}$.

For example, if $\epsilon = 0.05$, $n = 5000$, $p = 2^{-256}$, then degree $d$ need to be greater than $370.86$.

\end{proof}


\begin{lemma}
\label{lemma:randomgraph2}

For any $0 < \epsilon < 1$, there exist a degree $d$ such that a random $d$-regular bipartite graph $G=(L, R, E)$ with $|L| = |R| = n$ generated according to algorithm \ref{alg:randomgarph} satisfy the following property.

    \begin{itemize}
        \item Expansion: For every set $X \subset L$ with $|X| \ge \epsilon n$, if $Y$ is the set of neighbors of $X$ in $G$, then $|Y| \ge (1 - \epsilon)  n$ with high probability.
    \end{itemize}

\end{lemma}


\begin{proof}

We use the same trick as in lemma \ref{lemma:randomgraph}. Negating the statement, we can say that the randomly generated graph $G$ doesn't satisfy the expansion property if and only if 
for every $ S \subseteq L$, $|S| \ge \epsilon n$, $\exists M \subseteq R$, $|M| > \epsilon n$ such that there's no edge connecting between set $S$ and set $M$ with low probability. 
We bound the probability true as follows:

For every vertex $a \in L$ and every vertex $b \in R$, the probability that $a$ and $b$ are not connected in the random graph $G$ is:

$$P_1 = (\frac{n-1}{n})^{d}$$

For a set of vertices $S \subset L$ with $|S| = \epsilon n$, the probability that non of vertices in $S$ is connected to $b$ is:

$$P_2 = (P_1)^{\epsilon n} = (\frac{n-1}{n})^{d\epsilon n}$$

The probability that there exists at least $\epsilon n$ vertices in $R$ are not connected to any vertex in $S$ is:

\begin{align}
P_3 &= \binom{n}{\epsilon n} (P_2)^{\epsilon n} \nonumber \\
    &= \binom{n}{\epsilon n} (\frac{n-1}{n})^{d \epsilon^2 n^2} \nonumber \\
    &\le 2^{n H(\frac{\epsilon n}{n})} (\frac{n-1}{n})^{d \epsilon^2 n^2} 
    && \binom{n}{k} \le 2^{n H(\frac{k}{n})} \nonumber \\
    &= 2^{n H(\epsilon)} ((1 - \frac{1}{n})^{n})^{d \epsilon^2 n} \nonumber \\
    &\le 2^{n H(\epsilon)} (\frac{1}{e})^{d \epsilon^2 n} 
    && (1 - \frac{1}{x})^x \le \frac{1}{e} \text{ for } x \ge 1 \text{ (lemma \ref{lemma:(1-1x)x})} \nonumber \\
    &= (e^{ H(\epsilon)\ln 2 - d \epsilon^2})^{n} \nonumber \\
\end{align}

$P_3$ is the probability that a set $S$ in a randomly generated graph doesn't satisfy the expansion property. Suppose we want the failing probability be smaller than $p$, let $(e^{ H(\epsilon)\ln 2 - d \epsilon^2})^{n} < p$.
By rearranging the above equation, we have $ d > \frac{H(\epsilon) \ln 2 - \frac{\ln p}{n}}{\epsilon^2}$.

For example, if $\epsilon = 0.05$, $n = 5000$, $p = 2^{-256}$, then degree $d$ need to be greater than $93.60$.

Compared with lemma \ref{lemma:randomgraph}, lemma \ref{lemma:randomgraph2} produces a much tighter bound by weakening the expansion property. A graph satisfy the expansion property in lemma \ref{lemma:randomgraph2} may not satisfy the expansion property in lemma \ref{lemma:randomgraph}. There may exist a set $S \subset L$ in graph such that the expansion property fails. But lemma \ref{lemma:randomgraph2} guarantees that such set is hard to be found. Similar with hash functions, hash collision must exist somewhere, but this collision is hard to be found.

\end{proof}

\section{Reversed Linear Code}


\begin{figure}[h]
\centering
\begin{tikzpicture}


\draw [-stealth](0,0) node[anchor=east] {input} -- (1,0) node[anchor=west]{\textbf{.}};
\draw [-stealth](1.4,0) -- (5.4,0);

\draw (5.6,1) rectangle node{x} (6.2,-1.0);
\draw (5.6,-1.0) rectangle node{z} (6.2,-3.0);
\draw (5.6,-3.0) rectangle node{v} (6.2,-5.0);

\draw [-stealth](1.2,-0.2) -- node[below left] {$A$} (2.2,-1.8) node[anchor=north west]{\textbf{.}};
\draw [-stealth](2.6,-2) -- node[above] {Enc} (3.8,-2) node[anchor=west]{\textbf{.}};
\draw [-stealth](4.2,-2) -- (5.4,-2);
\draw [-stealth](4.0,-2.2) -- node[below left] {$B$} (5.4,-4);
\draw [-stealth](6.4,-2) -- (7.4,-2) node[anchor=west]{output};

\end{tikzpicture}
\caption{Linear Code}
\label{fig:lc}
\end{figure}

\begin{figure}[h]
\centering
\begin{tikzpicture}


\draw [stealth-](0,0) node[anchor=east] {output} -- (1,0) node[anchor=west]{\textbf{+}};
\draw [stealth-](1.45,0) -- (5.4,0);

\draw (5.6,1) rectangle node{x} (6.2,-1.0);
\draw (5.6,-1.0) rectangle node{z} (6.2,-3.0);
\draw (5.6,-3.0) rectangle node{v} (6.2,-5.0);

\draw [stealth-](1.2,-0.2) -- node[below left] {$A^T$} (2.2,-1.8);
\draw [stealth-](2.6,-2) node[anchor=east]{\textbf{+}} -- node[above] {Enc$^{-1}$} (3.8,-2) node[anchor=west]{\textbf{+}};
\draw [stealth-](4.25,-2) -- (5.4,-2);
\draw [stealth-](4.0,-2.2) -- node[below left] {$B^T$} (5.4,-4);
\draw [stealth-](6.4,-2) -- (7.4,-2) node[anchor=west]{input};

\end{tikzpicture}
\caption{Reversed Linear Code}
\label{fig:lc-rev}
\end{figure}